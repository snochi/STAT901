\documentclass[stat901]{subfiles}

%% ========================================================
%% document

\begin{document}

    \section{Probability Measures}

    \subsection{$\sigma$-fields}

    \begin{definition}{\textbf{$\sigma$-field} of Subsets of $\Omega$}
        Let $\Omega$ be a set and let $\mF\subseteq 2^{\Omega}$. We say $\mF$ is a \emph{$\sigma$-field} of subsets of $\Omega$ if
        \begin{enumerate}
            \item $\Omega\in\mF$;
            \item $A\in\mF$ implies $\Omega\setminus A\in\mF$; and\hfill\textit{closure under complement}
            \item $\left\lbrace A_n \right\rbrace^{\infty}_{n=1}\subseteq\mF$ implies $\bigcup^{\infty}_{n=1}A_n\in\mF$.\hfill\textit{closure under countable union}
        \end{enumerate}
        The elements of $\mF$ are called \emph{events} and the pair $\left( \Omega,\mF \right)$ is called a measurable space.
    \end{definition}

    \begin{example}{Simple $\sigma$-fields}
        Let $\Omega$ be a set.
        \begin{enumerate}
            \item The \emph{trivial $\sigma$-field} is $\left\lbrace \emptyset,\Omega \right\rbrace$.
            \item The power set $2^{\Omega}$ is also a $\sigma$-field.
            \item Given any $A\subseteq\Omega$, $\left\lbrace \emptyset, A, \Omega\setminus A, \Omega \right\rbrace$ is a $\sigma$-field.
            \item The collection of countable and co-countable sets,
                \begin{equation*}
                    \mF = \left\lbrace A\subseteq\Omega: \text{$A$ is countable or $X\setminus A$ is countable} \right\rbrace,
                \end{equation*}
                is a $\sigma$ field. To see this, let $\left\lbrace A_n \right\rbrace^{\infty}_{n=1}\subseteq\mF$. If every $A_n$ is countable, then so is $\bigcup^{\infty}_{n=1} A_n$. Hence $\bigcup^{\infty}_{n=1} A_n\in\mF$.

                On the other hand, if any $A_m$ is co-countable, then
                \begin{equation*}
                    X\setminus \left( \bigcup^{\infty}_{n=1} A_n \right) = X\setminus \left( A_m \cup \bigcup^{\infty}_{n=1, n\neq m} A_n \right) = \left( X\setminus \left( \bigcup^{\infty}_{n=1,n\neq m} A_n \right) \right) \setminus A_m \subseteq X\setminus A_m,
                \end{equation*}
                so that $X\setminus \left( \bigcup^{\infty}_{n=1}A_n \right)$ is co-countable. Thus $\bigcup^{\infty}_{n=1}A_n\in\mF$.
        \end{enumerate}
    \end{example}

    \rruleline

    \begin{example}{An Example of Important Non-$\sigma$-field}
        Consider $\Omega = \left[ 0,1 \right]$ and consider
        \begin{equation*}
            \begin{aligned}
                \mB_0 & = \left\lbrace \text{the finite unions of disjoint left-open-right-closed intervals} \right\rbrace \\
                      & = \left\lbrace \bigcup^{n}_{i=1} \left( a_i,b_i \right]: \left( a_i,b_i \right]\subseteq \left[ 0,1 \right], \left( a_i,b_i \right]\cap \left( a_j,b_j \right]=\emptyset\text{ for $i\neq j$} \right\rbrace.
            \end{aligned} 
        \end{equation*}
        Then $\mB_0$ is not a $\sigma$-field, since given $A_1 = \left( \frac{1}{2},1 \right], A_2 = \left( \frac{1}{3},\frac{1}{2} \right], \ldots, A_n = \left( \frac{1}{n+1},\frac{1}{n} \right], \ldots\subseteq \left[ 0,1 \right]$, we obtain:
        \begin{equation*}
            \bigcup^{\infty}_{n=1} A_n = \left[ 0,1 \right] \notin\mB_0.
        \end{equation*}
        However, we can check that the first two axioms ($\Omega\in\mB_0$ and closure under complement) hold for $\mB_0$ and that $\mF$ is closed under \textit{finite}, but not countable, union.

        $\mB_0$ is an example of \emph{field} (or \emph{algebra}) of subsets of $\Omega$.
    \end{example}

    \rruleline

    \begin{definition}{$\sigma$-field \textbf{Generated} by a Collection}
        Let $\Omega$ be a set and let $\mA\subseteq 2^{\Omega}$. If we let
        \begin{equation*}
            \sigma\left( \mA \right) = \bigcap^{}_{\substack{\mF\supseteq\mA:\\\text{$\mF$ is a $\sigma$-field}}} \mF,
        \end{equation*}
        then $\sigma\left( \mA \right)$ is a $\sigma$-field, called the $\sigma$-field \emph{generated} by $\mA$.
    \end{definition}

    \clearpage

    \begin{example}{Generating $\sigma$-fields}
        Let $\Omega$ be a set.
        \begin{enumerate}
            \item The trivial $\sigma$-field is generated by $\emptyset$.
            \item $\left\lbrace \emptyset, A,\Omega\setminus A, \Omega \right\rbrace$ is generated by $\left\lbrace \Omega\setminus A \right\rbrace$.
            \item The $\sigma$-field of countable and co-countable sets $\mF$ is generated by $\left\lbrace \left\lbrace \omega \right\rbrace \right\rbrace^{}_{\omega\in\Omega}$, the collection of singletons.
        \end{enumerate}
    \end{example}

    \rruleline

    \begin{example}{Borel $\sigma$-field on $\left( 0,1 \right]$}
        The \emph{Borel $\sigma$-field} on $\left( 0,1 \right]$ is the $\sigma$-field generated by $\mB_0$ (see Example 1.2).

        It can also be generated by $\left\lbrace \left[ a,b \right]\subseteq\Omega \right\rbrace, \left\lbrace \left( a,b \right)\subseteq\Omega \right\rbrace, \left\lbrace \left( a,b \right]\subseteq\Omega \right\rbrace, \left\lbrace \left[ a,b \right)\subseteq\Omega \right\rbrace$ (exercise).
    \end{example}

    \rruleline

    \begin{example}{General Borel $\sigma$-fields on Topological Spaces}
        Given any topological space $\left( \Omega,\tau \right)$, then the Borel $\sigma$-field on $\Omega$ is defined as $\sigma\left( \tau \right)$. If we let $\gamma$ to be the collection of closed sets then $\gamma$ also generates the Borel $\sigma$-fields on $\Omega$.
    \end{example}

    \rruleline

    \subsection{Probability Measure}

    \begin{definition}{\textbf{Probability Measure} on a Measurable Space}
        Let $\left( \Omega,\mF \right)$ be a measurable space. We say $\PP:\mF\to\left[ 0,\infty \right]$ is a \emph{measure} on $\mF$ if
        \begin{enumerate}
            \item $\PP\left( \emptyset \right) = 0$; and
            \item $\PP\left( \bigcupdot^{\infty}_{n=1} A_n \right) = \sum^{\infty}_{n=1} \PP\left( A_n \right)$.\hfill\textit{countable additivity}
        \end{enumerate}
        If $\PP$ satisfy in addition that $\PP\left( A \right)\in\left[ 0,1 \right]$ for all $A\in\mF$, then we say $\PP$ is a \emph{probability measure}.
    \end{definition}

    \begin{definition}{\textbf{Probability Space}}
        A \emph{probability space} is a triplet $\left( \Omega,\mF,\PP \right)$ such that
        \begin{enumerate}
            \item $\Omega$ is a set called \emph{sample space}, the set of all the possible results of a random experiments or observations;
            \item $\mF$ is a \emph{$\sigma$-field} of subsets of $\Omega$; and
            \item $\PP$ is a \emph{probability measure} on $\mF$.
        \end{enumerate}
    \end{definition}

    \begin{example}{Tossing a Coin}
        When we are tossing a coin $n$ times,
        \begin{equation*}
            \begin{aligned}
                \Omega & = \left\lbrace 0,1 \right\rbrace^n, \\
                \mF & = 2^{\Omega}, \\ 
                \PP\left( A \right) & = \frac{\left| A \right|}{2^n}, \hspace{1cm}\forall A\in\mF.
            \end{aligned} 
        \end{equation*}
    \end{example}

    \rruleline

    \begin{example}{Discrete Probability Space}
        Let $\Omega$ be a countable set and let $p:\Omega\to\left[ 0,1 \right]$ be such that
        \begin{equation*}
            \sum^{}_{\omega\in\Omega}p\left( \omega \right) = 1.
        \end{equation*}
        Then $\PP:2^{\Omega}\to\left[ 0,1 \right]$ by
        \begin{equation*}
            \PP\left( A \right) = \sum^{}_{\omega\in A}p\left( \omega \right), \hspace{1cm}\forall A\subseteq\Omega,
        \end{equation*}
        is a probability measure on $\left( \Omega,2^{\Omega} \right)$.

        We call $\left( \Omega,2^{\Omega},\PP \right)$ a \emph{discrete} probability space.
    \end{example}

    \rruleline

    \clearpage

    \begin{prop}{Properties of Probability Measures}
        Let $\left( \Omega,\mF,\PP \right)$ be a probability space. Then
        \begin{enumerate}
            \item $A\subseteq B$ implies $\PP\left( A \right)\leq \PP\left( B \right)$;\hfill\textit{monotonicity}

            \item $A\subseteq B$ implies $\PP\left( A \right) = \PP\left( B\setminus A \right) + \PP\left( B \right)$;\hfill\textit{excision}

            \item given any $\left\lbrace A_i \right\rbrace^{n}_{i=1}\subseteq\mF$,
                \begin{equation*}
                    \PP\left( \bigcup^{n}_{i=1}A_i \right) = \sum^{}_{i}\PP\left( A_i \right) - \sum^{}_{i<j} \PP\left( A_{i}\cap A_j \right) + \cdots + \left( -1 \right)^{n-1} \PP\left( A_1\cap\cdots\cap A_n \right);\eqno\text{\textit{inclusion-exclusion}}
                \end{equation*}

            \item for any increasing chain $\left( A_{n} \right)^{\infty}_{n=1}\in\mF^{\N}$, we have
                \begin{equation*}
                    \lim_{n\to\infty}\PP\left( A_n \right) = \PP\left( \bigcup^{\infty}_{n=1}A_n \right);\eqno\text{\textit{continuity from below}}
                \end{equation*}
            \item for any decreasing chain $\left( A_{n} \right)^{\infty}_{n=1}\in\mF^{\N}$, we have
                \begin{equation*}
                    \lim_{n\to\infty}\PP\left( A_n \right) = \PP\left( \bigcap^{\infty}_{n=1}A_n \right);\eqno\text{\textit{continuity from above}}
                \end{equation*}
                and
            \item for any $\left\lbrace A_n \right\rbrace^{\infty}_{n=1}\subseteq\mF$,
                \begin{equation*}
                    \PP\left( \bigcup^{\infty}_{n=1}A_n \right) \leq \sum^{\infty}_{n=1} \PP\left( A_n \right).\eqno\text{\textit{countable subadditivity (Bool's inequality)}}
                \end{equation*}
        \end{enumerate}
    \end{prop}

    \begin{proof}
        \begin{enumerate}
            \item Suppose $A\subseteq B$. Then $B\setminus A\in\mF$ as well with $B=A\cupdot B\setminus A$, so that $\PP\left( B \right) = \PP\left( A \right)+\PP\left( B\setminus A \right) \geq \PP\left( A \right)$.

            \item This is shown in (a).

            \item When $n=2$, we have $A_1\cup A_2 = \left( A_1\setminus \left( A_1\cap A_2 \right) \right)\cupdot A_2$, so that
                \begin{equation*}
                    \PP\left( A_1\cup A_2 \right) = \PP\left( A_1\setminus \left( A_1\cap A_2 \right) \right) + \PP\left( A_2 \right) = \PP\left( A_1 \right) + \PP\left( A_2 \right) - \PP\left( A_1\cap A_2 \right).
                \end{equation*}

                Assume that (c) holds for some $n\geq 2$. Then we note that
                \begin{flalign*}
                    && \PP\left( \bigcup^{n+1}_{i=1} A_i \right) & = \PP\left( \left( \bigcup^{n}_{i=1}A_i \right)\cup A_{n+1} \right) && \\ 
                    && & = \PP\left( \bigcup^{n}_{i=1}A_i \right) + \PP\left( A_{n+1} \right) - \PP\left( \left( \bigcup^{n}_{i=1}A_i \right)\cap A_{n+1} \right) && \\
                    && & = \PP\left( \bigcup^{n}_{i=1}A_i \right) + \PP\left( A_{n+1} \right) - \PP\left(  \bigcup^{n}_{i=1}\left(A_i \cap A_{n+1}\right) \right) && \\
                    && & = \left( \sum^{}_{i}\PP\left( A_i \right) - \sum^{}_{i<j} \PP\left( A_{i}\cap A_j \right) + \cdots + \left( -1 \right)^{n-1} \PP\left( A_1\cap\cdots\cap A_n \right)\right) && \\
                    && & + \PP\left( A_{n+1} \right) - \left( \sum^{}_{i} \PP\left( A_i\cap A_{n+1} \right)-\sum^{}_{i<j}\PP\left( A_i\cap A_j\cap A_{n+1} \right) + \cdots + \left( -1 \right)^{n+1}\PP\left( A_1\cap \cdots A_{n+1} \right) \right) && \\
                    && & = \sum^{n+1}_{i=1}\PP\left( A_{i} \right) - \sum^{n+1}_{i<j}\PP\left( A_i\cap A_j \right) + \cdots + \left( -1 \right)^{n+1}\PP\left( A_1\cap\cdots A_{n+1} \right).
                \end{flalign*}

            \item Let $\left( A_{n} \right)^{\infty}_{n=1}\in\mF^{\N}$ be an increasing chain. Define
                \begin{equation*}
                    B_n = A_n\setminus \bigcup^{n-1}_{i=1} A_i = A_n \setminus A_{n-1}
                \end{equation*}
                for all $n\in\N$. Then we observe that each $B_n\in\mF$ with $A_n = \bigcupdot^{n}_{i=1}B_i$, so that $\bigcup^{\infty}_{n=1}A_n = \bigcupdot^{\infty}_{i=1}B_i$ and
                \begin{equation*}
                    \lim_{n\to\infty} \PP\left( A_n \right) = \lim_{n\to\infty}\PP\left( \bigcupdot^{n}_{i=1} B_i \right) = \lim_{n\to\infty}\sum^{n}_{i=1} \PP\left( B_i \right) = \sum^{\infty}_{i=1}\PP\left( B_i \right) = \PP\left( \bigcupdot^{\infty}_{i=1}B_i \right) = \PP\left( \bigcup^{\infty}_{n=1}A_n \right),
                \end{equation*}
                as required.

            \item It suffices to note that, by taking $B_n = \Omega\setminus A_n$ for all $n\in\N$, we have
                \begin{equation*}
                    \lim_{n\to\infty} \PP\left( A_n \right) = \lim_{n\to\infty} 1-\PP\left( B_n \right) = 1-\PP\left( \bigcup^{\infty}_{n=1} B_n \right) = \PP\left( \Omega\setminus \bigcup^{\infty}_{n=1}B_n \right) = \PP\left( \bigcap^{\infty}_{n=1}A_n \right).
                \end{equation*}

            \item Let
                \begin{equation*}
                    B_n = A_n \setminus \bigcup^{n-1}_{i=1} A_i \subseteq A_n
                \end{equation*}
                for all $n\in\N$. Then $\bigcup^{\infty}_{n=1} A_n = \bigcupdot^{\infty}_{n=1}B_n$, so that
                \begin{equation*}
                    \PP\left( \bigcup^{\infty}_{n=1} A_n \right) = \PP\left( \bigcupdot^{\infty}_{n=1}B_n \right) = \sum^{\infty}_{n=1} \PP\left( B_n \right) \leq \sum^{\infty}_{n=1}\PP\left( A_n \right).
                \end{equation*}
        \end{enumerate}
    \end{proof}

    \subsection{Construction of Probability Measures}

    Given a $\sigma$-field $\mF$, it is hard to describe every elements in $\mF$. This means it is also hard to assign a probability measure $\PP$ on $\mF$.

    A natural idea to get around this is to first define $\PP$ on a subset of $\mF$and then extend it to the whole $\mF$.

    \begin{definition}{\textbf{Field} of Subsets of $\Omega$}
        We say $\mF_0\subseteq 2^{\Omega}$ is a \emph{field} of subsets of $\Omega$ if
        \begin{enumerate}
            \item $\emptyset\in\mF_0$;
            \item $A\in\mF_0$ implies $\Omega\setminus A\in\mF$; and\hfill\textit{closure under complement}
            \item $A,B\in\mF_0$ implies $A\cup B\in\mF_0$.\hfill\textit{closure under finite union}
        \end{enumerate}
    \end{definition}

    \np That is, a field is a subcollection of $2^{\Omega}$ that \textit{looks like} a $\sigma$-field that has closure under \textit{finite} union instead of countable union.

    \begin{definition}{\textbf{Premeasure} on a Field}
        Let $\mF_0$ be a field of subsets of $\Omega$. We say $\PP_0:\mF_0\to\left[ 0,\infty \right]$ is a \emph{premeasure} on $\mF$ if
        \begin{enumerate}
            \item $\PP_0\left( \emptyset \right) = 0$; and
            \item for any subcollection $\left\lbrace A_n \right\rbrace^{\infty}_{n=1}\subseteq\mF_0$ of disjoint elements,
                \begin{equation*}
                    \PP_0\left( \bigcupdot^{\infty}_{n=1}A_n \right) = \sum^{\infty}_{n=1}\PP_0\left( A_n \right).\eqno\text{\textit{countable additivity}}
                \end{equation*}
        \end{enumerate}
        If
        \begin{enumerate}
            \setcounter{enumi}{2}
            \item $\PP_0\left( \Omega \right)=1$
        \end{enumerate}
        as well, then we say $\PP_0$ is a \emph{probability} premeasure.
    \end{definition}

    \clearpage

    \begin{definition}{\textbf{Outer Measure} on a Set}
        We say $\PP^{*}:2^{\Omega}\to\left[ 0,\infty \right]$ is an \emph{outer measure} on $\Omega$ if
        \begin{enumerate}
            \item $\PP^{*}\left( \emptyset \right) = 0$; and
            \item for any $\left\lbrace A_n \right\rbrace^{\infty}_{n=1}\subseteq 2^{\Omega}$, $\PP^{*}\left( \bigcup^{\infty}_{n=1}A_n \right) \leq \sum^{\infty}_{n=1}\PP^{*}\left( A_n \right)$.\hfill\textit{countable subadditivity}
        \end{enumerate}

        Given $A\subseteq\Omega$, we say $A$ is \emph{$\PP^{*}$-measurable} if
        \begin{equation*}
            \PP^{*}\left( E \right) = \PP^{*}\left( E\cap A \right) + \PP\left( E\cap\left( \Omega\setminus A \right) \right),\hspace{1cm}\forall E\subseteq\Omega.\eqno\text{\textit{Caratheodory's criterion}}
        \end{equation*}
    \end{definition}

    \begin{theorem}{Extension Theorem}
        Let $\mF_0$ be a field of subsets of $\Omega$ and let $\PP_0:\mF_0\to\left[ 0,1 \right]$ be a probability premeasure on $\mF_0$. Then there exists a unique probability measure $\PP:\sigma\left( \mF_0 \right)\to\left[ 0,1 \right]$ such that $\PP|_{\mF_0} = \PP_0$.
    \end{theorem}

    \rruleline

    \np We split the proof of the theorem into few results.

    \begin{proof}[Proof of Existence]
        Let $\PP^{*}:2^{\Omega}\to\left[ 0,1 \right]$ be defined by
        \begin{equation*}
            \PP^{*}\left( A \right) = \inf \left\lbrace \sum^{\infty}_{n=1} \PP\left( A_n \right) : \left\lbrace A_n \right\rbrace^{\infty}_{n=1}\subseteq\mF_0, A\subseteq\bigcup^{\infty}_{n=1} A_n \right\rbrace,\hspace{1cm}\forall A\subseteq\Omega,
        \end{equation*}
        which is an outer measure on $\Omega$. Then by taking
        \begin{equation*}
            \mF = \left\lbrace A\subseteq\Omega : \text{$A$ is $\PP^{*}$-measurable} \right\rbrace,
        \end{equation*}
        we know that $\mF$ is a $\sigma$-field and $\PP = \PP^{*}|_{\mF}$ is a probability measure on $\left( \Omega,\mF \right)$ by Caratheodory's theorem.

        Now we check few claims.
        \begin{itemize}
            \item \textit{Claim 1. $\mF_0\subseteq\mF$.}

                \begin{subproof}
                    Let $A\in\mF_0$. For any $E\subseteq\Omega$, we desire to show
                    \begin{equation*}
                        \PP^{*}\left( E \right) = \PP^{*}\left( E\cap A \right) + \PP^{*}\left( E\cap \left( \Omega\setminus A \right) \right).
                    \end{equation*}
                    For any $\epsilon>0$, let $\left\lbrace A_n \right\rbrace^{\infty}_{n=1}\subseteq\mF_0$ be such that, and
                    \begin{equation*}
                        B_n = A\cap A_n, C_n = A\cap\left( \Omega\setminus A_n \right),\hspace{1cm}\forall n\in\N.
                    \end{equation*}
                    Then
                    \begin{equation*}
                        \begin{aligned}
                            E\cap A & \subseteq \bigcup^{\infty}_{n=1} B_n \\
                            E\cap\left(\Omega\setminus A\right) & \subseteq \bigcup^{\infty}_{n=1} C_n
                        \end{aligned} .
                    \end{equation*}
                    This means
                    \begin{equation*}
                        \begin{aligned}
                            \PP^{*}\left( E\cap A \right)&\leq\PP^{*}\left( \bigcup^{\infty}_{n=1}B_n \right) \leq \sum^{\infty}_{n=1} \PP\left( B_n \right) \\
                            \PP^{*}\left( E\cap\left( \Omega\setminus A \right)\right)&\leq\PP^{*}\left( \bigcup^{\infty}_{n=1}C_n \right) \leq \sum^{\infty}_{n=1} \PP\left( C_n \right)
                        \end{aligned} .
                    \end{equation*}
                    Thus
                    \begin{equation*}
                        \PP^{*}\left( E\cap A \right) + \PP^{*}\left( E\cap \left( \Omega\setminus A \right) \right) \leq \sum^{\infty}_{n=1} \PP\left( B_n \right) + \sum^{\infty}_{n=1} \PP\left( C_n \right) = \sum^{\infty}_{n=1}\PP\left( A_n \right) \leq \PP^{*}\left( E \right)+\epsilon.
                    \end{equation*}
                    Since this holds for all $\epsilon>0$, it follows $\PP^{*}\left( E \right)\geq\PP^{*}\left( E\cap A \right)+\PP^{*}\left( E\cap\left( \Omega\setminus A \right) \right)$.

                    The other direction is trivial, as union of any covers for $E\cap A, E\cap\left( \Omega\setminus A \right)$, respectively, is a cover for $E$. Hence we have shown the desired equality, which imply $A\in\mF$. Thus $\mF_0\subseteq\mF$.
                \end{subproof}
        \end{itemize} 
        A consequence of Claim 1 is that, $\mF\supseteq\sigma\left( \mF_0 \right)$. Since $\PP = \PP^{*}|_{\mF}$ is a probability measure on the $\sigma$-field $\mF$, it follows it is also a probability measure on $\sigma\left( \mF_0 \right)$. Moreover,
        \begin{equation*}
            \PP_0\left( A \right) = \PP\left( A \right),\hspace{1cm}\forall A\in\mF_0.
        \end{equation*}
        This means $\PP$ is an extension of $\PP_0$ on $\sigma\left( \mF_0 \right)$.
    \end{proof}

    \begin{definition}{\textbf{$\pi$-system}, \textbf{$\lambda$-system} of Subsets of $\Omega$}
        We say $\Pi\subseteq 2^{\Omega}$ is a \emph{$\pi$-system} of subsets of $\Omega$ if
        \begin{equation*}
            \forall A,B\in\Pi\left[ A\cap B\in\Pi \right].\eqno\text{\textit{closure under intersection}}
        \end{equation*}

        We say $\Lambda\subseteq 2^{\Omega}$ is a \emph{$\lambda$-system}, if
        \begin{enumerate}
            \item $\emptyset\in\Lambda$;
            \item $A\in\Lambda$ implies $\Omega\setminus A\in\Lambda$; and\hfill\textit{closure under complement}
            \item for any collection $\left\lbrace A_n \right\rbrace^{\infty}_{n=1}\subseteq 2^{\Omega}$ of disjoint subsets of $\Omega$, $\bigcupdot^{\infty}_{n=1}A_n\in\Lambda$.\hfill\textit{closure under countable disjoint union}
        \end{enumerate}
    \end{definition}

    \begin{prop}{}
        Let $\mF\subseteq 2^{\Omega}$. Then
        \begin{equation*}
            \text{$\mF$ is a $\sigma$-field} \iff \text{$\mF$ is a $\pi$-system and a $\lambda$-system}.
        \end{equation*}
    \end{prop}

    \begin{proof}
        ($\implies$) This direction is more-or-less trivial.

        ($\impliedby$) It suffices to show that $\mF$ is closed under countable union. Let $\left\lbrace A_n \right\rbrace^{\infty}_{n=1}\subseteq 2^{\Omega}$. Define
        \begin{equation*}
            B_n = A_n \bigcap^{n-1}_{i=1} \left( \Omega\setminus A_i \right).
        \end{equation*}
        Then note that each $B_n\in\mF$, as $\mF$ is closed under complement (as $\mF$ is a $\lambda$-system) and closed under intersection (as $\mF$ is a $\pi$-system). 

        By definition $\left\lbrace B_n \right\rbrace^{\infty}_{n=1}$ is a collection of pairwise disjoint subsets of $\Omega$, so by the fact that $\mF$ is a $\lambda$-system,
        \begin{equation*}
            \bigcup^{\infty}_{n=1} A_n = \bigcupdot^{\infty}_{n=1} B_n \in \mF.
        \end{equation*}
    \end{proof}

    \begin{prop}{}
        Let $\Lambda\subseteq 2^{\Omega}$ be a $\lambda$-system. If $A,B\subseteq\Omega$ are such that $A,A\cap B\in\Lambda$, then $A\cap\left( \Omega\setminus B \right)\in\Lambda$.
    \end{prop}

    \begin{proof}
        Since $\Lambda$ is closed under complement, $\Omega\setminus A\in\Lambda$. Since $\Omega\setminus A, B$ are disjoint, it follows $\left( \Omega\setminus A \right)\cup\left( A\cap B \right)\in\Lambda$. By taking its complement
        \begin{equation*}
            A\cap \left( \Omega\setminus B \right) = \Omega\setminus\left( \left( \Omega\setminus A \right)\cup \left( A\cap B \right) \right) \in \Lambda.
        \end{equation*}
    \end{proof}

    \begin{theorem}{$\pi-\lambda$ Theorem}
        Let $\Pi$ be a $\pi$-system and let $\Lambda$ be a $\lambda$-system. If $\Pi\subseteq\Lambda$, then $\sigma\left( \Pi \right)\subseteq\Lambda$.        
    \end{theorem}

    \begin{proof}
        Define
        \begin{equation*}
            \lambda\left( \Pi \right) = \bigcap^{}_{} \left\lbrace \mL\cap\Pi:\text{$\mL$ is a $\lambda$-system containing $\Pi$} \right\rbrace.
        \end{equation*}
        It is a routine task to show that $\lambda\left( \Pi \right)$ is also a $\lambda$-system containing $\Pi$.

        For any $A\subseteq\Omega$, define
        \begin{equation*}
            \mC_A = \left\lbrace B\subseteq\Omega : A\cap B\in\lambda\left( \Pi \right) \right\rbrace.
        \end{equation*}
        \begin{itemize}
            \item \textit{Claim 1. $\mC_A$ is a $\lambda$-system containing $\lambda\left( \Pi \right)$.}

                \begin{subproof}
                    Let $A\in\lambda\left( \Pi \right)$ and we check three things.
                    \begin{enumerate}
                        \item $A\cap\Omega = A \in \lambda\left( \Pi \right)$, which means $\Omega\in\mC_A$.
                        \item Given $B\in\mC_A$, then $\lambda\left( \Pi \right)$ is a $\lambda$-system containing both $A, A\cap B$. Then we know that $A\cap \left( \Omega\setminus B \right)\in\lambda\left( \Pi \right)$. It follows $\Omega\setminus B\in\lambda\left( \Pi \right)$.
                        \item If $\left\lbrace B_n \right\rbrace^{\infty}_{n=1}\subseteq\mC_A$ is a collection of disjoint sets in $\mC_A$, then $A\cap B_1, A\cap B_2, \ldots$ are in $\lambda\left( \Pi \right)$ and are disjoint. By taking the union of $A\cap B_n$'s,
                            \begin{equation*}
                                \bigcup^{\infty}_{n=1} A\cap B_n = A \cap \bigcup^{\infty}_{n=1} B_n \in\lambda\left( \Pi \right).
                            \end{equation*}
                            Thus $\bigcup^{\infty}_{n=1} B_n \in \lambda\left( \Pi \right)$.
                    \end{enumerate}
                    Moreover, if $A\in\Pi$, then for any $B\in\Pi$,
                    \begin{equation*}
                        A\cap B\in\Pi\subseteq\lambda\left( \Pi \right).
                    \end{equation*}
                    Then $B\in\mC_A$. Hence $\Pi\subseteq\mC_A$. Since $\mC_A$ is a $\lambda$-system, $\lambda\left( \Pi \right)\subseteq\mC_A$.
                \end{subproof}
        \end{itemize} 

        Now assume $A\in\Pi, B\in\lambda\left( \Pi \right)$. Then $B\in\mC_A$, so $A\cap B\in\lambda\left( \Pi \right)$. This also means $A\in\mC_B$. Since this holds for all $A\in\Pi$, we have
        \begin{equation*}
            B\in\lambda\left( \Pi \right) \implies \Pi\subseteq\mC_B \implies \lambda\left( \Pi \right)\subseteq\mC_B.
        \end{equation*}
        Therefore, for any $A,B\in\lambda\left( \Pi \right)$, $A\in\mC_B$. Hence $A\cap B\in\lambda\left( \Pi \right)$, which means $\lambda\left( \Pi \right)$ is a $\pi$-system, so that it is a $\sigma$-field. As a result,
        \begin{equation*}
            \Pi \subseteq\sigma\left( \Pi \right)\subseteq\lambda\left( \Pi \right)\subseteq\Lambda.
        \end{equation*}
    \end{proof}

    \begin{cor}{}
        Let $\Pi\subseteq 2^{\Omega}$ be a $\pi$-system and suppose that two probability measures $\PP_1,\PP_2$ agree on $\Pi$. Then they agree on $\sigma\left( \Pi \right)$.
    \end{cor}	

    \begin{proof}
        Let
        \begin{equation*}
            \Lambda = \left\lbrace A\in\Pi: \PP_1\left( A \right) = \PP_2\left( A \right) \right\rbrace.
        \end{equation*}

        \begin{subproof}[Claim 1]
            \textit{$\Lambda$ is a $\lambda$-system.}

            Note that $\PP_1\left( \emptyset \right) = 0 = \PP_2\left( \emptyset \right)$ so that $\emptyset\in\Lambda$.

            Suppose that $A\in\Lambda$. Then
            \begin{equation*}
                \PP_1\left( \Omega\setminus A \right) = 1-\PP_1\left( A \right) = 1-\PP_2\left( A \right) = \PP_2\left( \Omega\setminus A \right),
            \end{equation*}
            so that $\Omega\setminus A\in\Lambda$.

            Let $\left\lbrace A_n \right\rbrace^{\infty}_{n=1}\subseteq\Lambda$ be a subcollection of disjoint sets. Then
            \begin{equation*}
                \PP_1\left( \bigcup^{\infty}_{n=1}A_n \right) = \sum^{\infty}_{n=1} \PP_1\left( A_n \right) = \sum^{\infty}_{n=1} \PP_2\left( A_n \right) = \PP_2\left( \bigcup^{\infty}_{n=1}A_n \right),
            \end{equation*}
            so that $\bigcup^{\infty}_{n=1}A_n\in\Lambda$.

            \hfill\textit{(Done with Claim 1)}
        \end{subproof}

        So $\Lambda$ is a $\lambda$-system containing $\Pi$. By the $\pi-\lambda$ theorem, $\Lambda\supseteq\sigma\left( \Pi \right)$. Thus $\Lambda = \sigma\left( \Pi \right)$, as required.
    \end{proof}

    \np The uniqueness part of the theorem follows immediately from Corollary 1.5.1 and the fact that a field is a $\pi$-system.

    \begin{example}{Lebesgue Measure on $\left( 0,1 \right]$}
        Let $\Omega = \left( 0,1 \right]$ and let
        \begin{equation*}
            \mB_0 = \left\lbrace \bigcupdot^{n}_{k=1} I_k : \text{$I_1,\ldots,I_k\subseteq\left( 0,1 \right]$ are disjoint intervals} \right\rbrace.
        \end{equation*}
        Then $\mB_0$ is a field.

        Define $\lambda:\mB_0\to\left[ 0,1 \right]$ such that
        \begin{equation*}
            \lambda\left( \bigcupdot^{n}_{k=1} I_k \right) = \sum^{n}_{k=1} \lambda\left( I_k \right)
        \end{equation*}
        for all $\bigcupdot^{n}_{k=1} I_k\in\mB_0$, where $\lambda\left( I_k \right) = b_k-a_k$ for any interval $I_k$ with endpoints $a_k<b_k$. Then $\lambda$ is a probability premeasure on $\mB_0$.

        So by the extension theorem, there exists a unique probability measure $\overline{\lambda}:\mB\left( \left( 0,1 \right] \right)\to\left[ 0,1 \right]$ on $\sigma\left( \mB_0 \right) = \mB\left( \left( 0,1 \right] \right)$ that extends $\lambda$.

        We call $\overline{\lambda}$ the \emph{Lebesgue measure} on $\left( 0,1 \right]$.
    \end{example}

    \rruleline

    \begin{definition}{\textbf{Complete} Measure}
        Let $\left( \Omega,\mF \right)$ be a measurable space. We say $\PP$ is \emph{complete} probability measure on $\left( \Omega,\mF \right)$ if $\PP$ is a probability measure with
        \begin{equation*}
            \forall A\in\mF\left[ \PP\left( A \right) = 0 \implies \forall B\subseteq A\left[ B\in\mA \right] \right].
        \end{equation*}
        In this case, we say $\left( \Omega,\mF,\PP \right)$ is a \emph{complete} probability space.
    \end{definition}

    \np If $\left( \Omega,\mF,\PP \right)$ is a complete probability space, then for $A\subseteq\Omega$, if there is $B\in\mF$ such that
    \begin{equation*}
        A\triangle B \subseteq C
    \end{equation*}
    for some $C\in\mF$ with $\PP\left( C \right) = 0$, then $A\in\mF$ with $\PP\left( A \right) = \PP\left( B \right)$.

    \begin{prop}{}
        Let $\left( \Omega,\mF,\PP \right)$ be a probability space. Then there exists a unique complete probability space $\left( \Omega,\mF',\mP' \right)$ such that $\mF\subseteq\mF'$ and $\PP'|_{\mF} = \PP$.
    \end{prop}

    \begin{proof}
        Let
        \begin{equation*}
            \mM = \left\lbrace A\subseteq\Omega:\text{$A$ is $\PP^{*}$-measurable} \right\rbrace,
        \end{equation*}
        where $\PP^{*}:2^{\Omega}\to\left[ 0,1 \right]$ is the outer measure extending $\PP$. Then recall that $\PP^{*}$ is a probability measure on $2^{\Omega}$.

        We are going to show that $\PP^{*}|_{\mM}$ is a complete measure on $\left( \Omega,\mM \right)$. So let $A\in\mM$ be such that $\PP^{*}\left( B \right) = 0$ and let $A\subseteq B$. We must show that $A$ is $\PP^{*}$-measurable, so let $E\subseteq\Omega$.

        Then
        \begin{equation*}
            \PP^{*}\left( E\cap A \right) + \PP^{*}\left( E\cap\left( \Omega\setminus A \right) \right)\leq \PP^{*}\left( B \right) + \PP^{*}\left( E \right) = \PP^{*}\left( E \right).
        \end{equation*}
        The other direction is trivial, as usual.

        Then $\PP^{*}\left( A \right) = 0$ follows from the monotonicity of outer measures.
    \end{proof}

























    
    
    
    

\end{document}
